\documentstyle{article}
 \topmargin=0pt
 \oddsidemargin=0pt \evensidemargin=0pt %
 \textwidth=14cm                         %
 \textheight=21cm                        %
                                         %
 \title{This is a sample of \LaTeX}
 \author{Bigeyes(\tt{chencs@263.net})\\
         Department of Mathematics\\     %
         USTC, Anhui, Hefei, 230026      %
                                         %
 \and
         A\_A                            %
 \thanks{Supported by ...}
         \\                              %
         BBS of USTC\\                   %
         bbs.ustc.edu.cn[202.38.64.3]}   %
 \date{Mar. 1, 2009}

 \begin{document}

 \maketitle
                                          \begin{abstract}                        %
 This is mini-page defined for abstract, %
 you only write your abstract in it. If %
 you want to shows keywords, maybe you   %
 should use:                             %

 {\bf Keywords: }\LaTeX, example         %
 \end{abstract}                          %
                                         %
 \section{The very beginning}            %
 This is the first section of your       %
 article. You may find every first       %
 paragraph of your section, subsection, %
 chapter or ... always has no            %
 ``parindent'' at the beginning.         %

 This is the second paragrph, you can    %
 find this has parindent at the          %
 beginning. If you want to show          %
 parindent at first paragraph too,       %
 do as the first paragraph I showed      %
 in the next section.                    %
                                         %
 \section{The 2nd step}                  %
 \hskip \parindent                       %
 This is the second section. In this     %
 first paragraph, I use `hskip' to       %
 get the first parindent. Maybe you      %
 can get this effect by another way.     %
                                         %
 \subsection{Sub-sect of 2}              %
 this                                    %
                                         %
 \subsection*{\S 2.2 Another sub of 2}   %
 this                                    %
                                         %
 \section{Conclusion}                    %
 I think you have know \TeX well now.    %
 I want to show you how to use           %
 bibliography. In the article, you       %
 can use as ``see \cite{texbook}''.      %
                                         %
 \begin{thebibliography}{0}


 \bibitem{texbook} Donald~E.~Knouth, ``The \TeX book'',
 Addison-Wesley, 1984
 \bibitem{lamport} L.\ Lamport, ``\LaTeX:
         A Document Preparation System'',
         Addison-Wesley, 1994
 \bibitem{companion} M.~Goossens, F.~Millelbach,
         and A.~Samarin, ``The \LaTeX\ Companion'',
         Addison--Wesley, 1994
 \end{thebibliography}

 \end{document} 