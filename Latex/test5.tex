\documentclass[UTF8]{ctexart}
\usepackage{geometry}
\geometry{left=5.0cm,right=2.0cm, top=2.0cm, bottom=2.0cm}
\title{这是我的第一份\LaTeX文档}
\author{任振华\\2452503780@qq.com\\计算机学院\\数据科学与大数据专业\\UESTC,Chengdu,Sichuan,611731}
\date{\today}

\bibliographystyle{plain}
\begin{document}
\maketitle
\section{第一段}
Most people would probably agree that many individual consumer adverts function on the level of the daydream.By picturing quite unusually happy and glamorous people whose success in either career of sexual terms, or both, is obvious, adverts construct an imaginary world in which the reader is able to make come true those desires which remain unsatisfied in his or her everyday life.
\\
\textbf{\begin{flushleft}写一段个人简介\end{flushleft}}
\subsection{第一小节}
\begin{center}\underline{An advert for a science fiction magazine is unusually explicit about this.}\end{center} In addition to the primary use value of the magazine, the reader is promised access to a wonderful universe through the product—access to other mysterious and tantalizing worlds and epochs, the realms of the imagination. When studying advertising, it is therefore unreasonable to expect readers to decipher adverts as factual statements about reality. Most adverts are just too meagre in informative content and too rich in emotional suggestive detail to be read literally. If people read then literally, they would soon be forced to realize their error when the glamorous promises held out by the adverts didn’t materialize.
Knuth 在发明 TeX 的时候,根本没有考虑到还要处理中文字符(以及其它许多亚洲字符)──它发明 TeX 的目的就是为了排版它的巨著《计算机算法艺术》。尽管这样,Knuth 却采用了一种先进的设计思想,从底层留下了扩展接口,并将其全面公开。这样,当时他本人没有实现的一些功能,就可以通过宏包的形式加以扩展实现。这就是为什么到现在为止 30多年过去了,TeX 在底层还几乎没有改动(只发现两处小错误)。
\subsubsection{第三段}
The average consumer is not surprised that his purchase of the commodity does not redeem the promise of the advertisement, for this is what he is used to in life: the individual’s pursuit of happiness and success is usually in vain. But the fantasy is his to keep; in his dream world he enjoys a “future endlessly deferred”.
\section{ww}
\textbf{\emph{Latex中处理中文,需要用一个叫做 CJK 的宏包(宏包就是预先定义了一些命令及格式的一个文档,学过 C 语言的同学都应该了解的)。}}
The Estivalia advert is quite explicit about the fact that advertising shows us not reality, but a fantasy; it does so by openly admitting the daydream but in a way that insists on the existence of a bridge linking daydream to reality—Estivalia, which is “for daydream believers”, those who refuse to give up trying to make the hazy ideal of natural beauty and harmony come true.
CJK是由Werner Lemberg开发的支持中、日、韩、英文字的宏包。CJK的特点是不需要象CCT那样预处理,支持PDFLatex和Type1字体,因此得到越来越多中国TeXer的喜爱,逐渐成为中文LaTeX的主流。
If adverts function on the daydream level, it clearly becomes in adequate to merely condemn advertising for channeling readers’ attention and desires towards an unrealistic, paradisiacal nowhere land. Advertising certainly does that, but in order for people to find it relevant, the utopia visualized in adverts must be linked to our surrounding reality by a casual connection
hots\_name
\end{document}