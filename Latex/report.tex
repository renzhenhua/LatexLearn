\documentclass[UTF8]{ctexart}
\usepackage{amssymb}
\usepackage{amsmath}
\usepackage{CJK}
\usepackage{geometry}
\CTEXsetup[format={\Large\bfseries}]{section}
\geometry{left=5.0cm,right=2.0cm, top=2.0cm, bottom=2.0cm}
\title{这是任振华的第一份\LaTeX文档}
\author{任振华\\2452503780@qq.com\\计算机学院\\数据科学与大数据专业\\UESTC,Chengdu,Sichuan,611731}
\date{\today}

\begin{document}
\maketitle
\begin{abstract}
A user identity anonymous is am important propetry.\\
{\bf Keywords: }\LaTeX sdfsd
\end{abstract}
\textbf{\begin{flushleft}
{\section{Introduction}}
\end{flushleft}}
\noindent
In 2004,Zhu and Ma [1] proposed an authentication scheme with anonymity for wireless communication environ-ments. Later, Lee et al. [2] showed several security flaws of Zhu-Ma's scheme and then improved it.However, in2008,Wu et al.[3] showed that both Zhu-Ma's scheme and Lee et al.'s scheme still cannot provide anonymity andthen proposed an improvement to preserve anonymity. Nevertheless Zeng et al.[4] and Lee et al.[5] showed that Wuet al.'s scheme also cannot provide anonymity,respectively.\\
 \indent
In 2011,Kang et al. [7] proposed an improved user authentication scheme based on both Wu et al.'s and Wei etal.'s scheme[3], [6] that guarantees strong user anonymity in wireless communications. However, this letter shows that the Kang et al.'s improved scheme also cannot provide user anonymity as they claimed.

\textbf{\begin{flushleft}
{\section{Review of Kang et al.s Scheme}}
\end{flushleft}}

\begin{table}[!hbp]
\caption{Notations}
\centering
\begin{tabular}{|l|l|}
  \hline
  % after \\: \hline or \cline{col1-col2} \cline{col3-col4} ...
  $ HA $& Home Agent of a mobile user \\
  $ FA $& Foreign Agent of the network \\
  $ MU $& Mobile User \\
  $ PW_{MU} $& A password of MU \\
  $ N $& A strong secret key of HA \\
  $ ID_A $& Identity of an entity A \\
  $ T_A $& Timestamp generated by an entity A \\
  $ Cert_A $& Certiface of an entity A \\
  $ (X)_{K} $& Encryption of message X using symmetric key K \\
  $ E_{P_{A}}(X) $& Encryption of message X using public key A \\
  $ S_{S_{A}} $&Encryption of message X using private key A  \\
  $ h(-) $& A one-way hash function \\
  $ \| $& Concatenation \\
  $ \oplus $& Bitwise exclusive-or opertaion\\
  \hline
\end{tabular}
\end{table}


\subsection{Initial Phase}
When an MU registers
\begin{equation}
PW_{MU}=h(N\|ID_{MU})
\end{equation}
\begin{equation}
r_1=h(N\|ID_{HA})
\end{equation}
\begin{equation}
r_2=h(N\|TD_{MU})\oplus{ID_{NA}}\oplus{ID_{MU}}
\end{equation}

\subsection{First Phase}
\begin{equation}
n=h(T_{MU}\|r_1)\oplus r_2\oplus PW_{MU}
\end{equation}
\begin{equation}
L=h(T_{MU}\oplus PW_{MU})
\end{equation}
\begin{equation}
ID_{MU}=h(T_{MU}\|h(N||ID_{HA}))\oplus n\oplus ID_{HA}
\end{equation}
\begin{equation}
\begin{aligned}
k&=h(h(h(h(N\|ID_{MU}))\|x\|x_0)\\
 &=h(h(PW_{MU}))\|x\|x_0
\end{aligned}
\end{equation}

\subsection{Second Phase}
\begin{equation}
k=h(h(h(h(N\|ID_{MU}))\|x\|x_{i-1}
\end{equation}

\section{Anonymity Problem of Kang et al.s Scheme}
\begin{equation}
\begin{aligned}
n'=&h(T'_{MU}\|r_{1}\oplus PW'_{MU}\\
  =&h(T'_{MU}\|h(N\|ID'_{MU})\oplus ID_{HA}\\
   &  \oplus ID'_{MU}\oplus PW'_{MU} \\
  =&h(T'_{MU}\|r_{1})\oplus h(N\|ID'_{MU}\oplus ID_{HA})\\
   &  \oplus ID'_{MU}\oplus h(N\|ID'_{MU})\\
  =& h(T'_{MU}\|r_{1})\oplus ID_{HA}\oplus ID_{MU}
\end{aligned}
\end{equation}

\begin{equation}
\begin{aligned}
ID'_{MU} =& n'\oplus(T'_{MU}\|r_{1})\\
         =& h(T'_{MU}\|r_{1})\oplus ID_{HA}\oplus ID'_{MU}\\
          & \oplus ID_{HA}\oplus h(T'_{MU}\|r_{1})\\
         =& ID'_{MU}
\end{aligned}
\end{equation}

\section{Conclusions\\Acknowledgements}
\end{document}
